\documentclass[10pt,a4paper]{article}
\usepackage[utf8]{inputenc}
\usepackage[spanish]{babel}
\usepackage{amsmath}
\usepackage{amsfonts}
\usepackage{amssymb}
\usepackage{makeidx}
\usepackage{graphicx}
\usepackage{cite} % para contraer referencias
\usepackage{fourier}
\usepackage{xcolor}
\usepackage{hyperref}
 
\usepackage[bottom]{footmisc}
\usepackage[left=2cm,right=2cm,top=2cm,bottom=2cm]{geometry}
\title{Cronograma Universo Medible II - 2020}


\author{\textbf{Victor M. Santos}\thanks{victorhugo\_m09@hotmail.com}, \textbf{M.Tarazona-Alvarado}\thanks{miguelta281@gmail.com}, \textbf{J. Pisco-Guabave} \thanks{jhojavi@gmail.com}. \\ Grupo Halley, \\ Universidad Industrial de Santander, Bucaramanga, Colombia.}


\date{ }


\begin{document}

\maketitle

Universo medible II consta de 16 sesiones con una duración de 2 horas cada una, cuenta con material teórico y actividades didácticas. La población objetiva del proyecto son estudiantes de grado décimo y once en el sistema educativo Colombiano. 

\tableofcontents

\section{Contenido}
\subsection{Sesión 1}
\begin{itemize}
 \item Cuerpos celestes
 \item Astronomía de posición
 \item Definiciones
 \begin{itemize}
  \item Esfera celeste
  \item Ecuador celeste
  \item Eclíptica
  \item Cenit y nadir
  \item Coordenadas celestes
 \end{itemize}
\end{itemize}

\subsection{Sesión 2}
\begin{itemize}
 \item Rotación
 \begin{itemize}
  \item El día y la noche
  \item ¿Tierra esférica?
 \end{itemize}
 \item Traslación
 \begin{itemize}
  \item Solsticios y equinoccios
 \end{itemize}
 \item Nutación
 \item Precesión
 \item Bamboleo de Chandler
\end{itemize}

\subsection{Sesión 3}
\begin{itemize}
 \item Unidades astronómicas
 \item Distancias y tamaños en el Sistema Solar
 \item Paralaje
\end{itemize}

\subsection{Sesión 4}
\begin{itemize}
 \item Constelaciones
 \item Carta celeste
 \item Aplicaciones usadas en la astronomía
\end{itemize}

\subsection{Sesión 5}
\begin{itemize}
 \item Propiedades de la Luz
 \begin{itemize}
  \item Reflexión
  \item Refracción
  \item Dispersión
  \item Difracción
 \end{itemize}
\end{itemize}

\subsection{Sesión 6}
\begin{itemize}
 \item Coordenadas geográficas 
 \item Coordenadas celestes
 \begin{itemize}
  \item Coordenadas horizontales
  \item Coordenadas ecuatoriales
  \item Coordenadas eclipticas
 \end{itemize}
\end{itemize}



\section{Sesión 0: Bienvenida}
En esta sesión se hace una presentación general de Universo Medible II y se da una charla divulgativa sobre astronomía general. \\ 

\href{https://github.com/miguelta281/Universo_Medible_II/blob/master/Presentaciones/Sesiones/Bienvenida/Bienvenida.pdf}{\underline{Presentación}} \\

\href{https://github.com/miguelta281/Universo_Medible_II/blob/master/Presentaciones/Sesiones/Bienvenida/Charla_bienvenida/Fetu.pdf}{\underline{Charla}} 
\section{Sesión 1: Nociones fundamentales}
En esta sesión se abarcan contenidos fundamentales para la compresión del término \textit{astronomía} como ciencia que tiene por objeto de estudio el comportamiento de los astros. Definiendo, sutilmente, características de los diferentes cuerpos que se pueden estudiar, las convenciones usadas para estudiar su movimiento aparente y las primeras interpretaciones del cosmos.\\

\href{https://github.com/miguelta281/Universo_Medible_II/blob/master/Presentaciones/Sesiones/Sesion_1/sesion_1.pdf}{\underline{Presentación}} \\

\href{https://github.com/miguelta281/Universo_Medible_II/blob/master/Organizacion/Planes/Sesion_1/Plan_sesion_1.pdf}{\underline{Plan}} \\

\section{Sesión 2: Movimientos de la Tierra}
En esta sesión se estudian los movimientos de la Tierra. El uso de experimentos y maquetas asumen el rol principal en la comprensión de la temática. \\

\href{https://github.com/miguelta281/Universo_Medible_II/blob/master/Presentaciones/Sesiones/Sesion_2/sesion_2.pdf}{\underline{Presentación}} \\

\href{https://github.com/miguelta281/Universo_Medible_II/blob/master/Organizacion/Planes/Sesion_2/Plan_sesion_2.pdf}{\underline{Plan}} \\

\end{document}