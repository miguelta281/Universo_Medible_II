\documentclass[10pt,a4paper]{article}
\usepackage[utf8]{inputenc}
\usepackage[spanish]{babel}
\usepackage{amsmath}
\usepackage{amsfonts}
\usepackage{amssymb}
\usepackage{makeidx}
\usepackage{graphicx}
\usepackage{cite} % para contraer referencias
\usepackage{fourier}
\usepackage{xcolor}
\usepackage{hyperref}
 
\usepackage[bottom]{footmisc}
\usepackage[left=2cm,right=2cm,top=2cm,bottom=2cm]{geometry}
\title{Cronograma Universo Medible II - 2020}


\author{\textbf{Victor M. Santos}\thanks{victorhugo\_m09@hotmail.com}, \textbf{M.Tarazona-Alvarado}\thanks{miguelta281@gmail.com}, \textbf{J. Pisco-Guabave} \thanks{jhojavi@gmail.com}. \\ Grupo Halley de Astronomía y Ciencias Aeroespaciales. \\ Universidad Industrial de Santander. Bucaramanga, Colombia.}


\date{ }


\begin{document}
\maketitle

Universo medible II consta de 16 sesiones con una duración de 2 horas cada una, cuenta con material teórico y actividades didácticas. La población objetiva del proyecto son estudiantes de grado décimo y once en el sistema educativo Colombiano. 

\tableofcontents
\section{Contenidos}
\begin{itemize}
\item Nociones fundamentales
 \begin{itemize}
  \item Cuerpos celestes 
  \item Astronomía de posición
  \item Definiciones (Esfera celeste, ecuador celeste, ecliptica, cenit, nadir, coordenadas celestes)
 \end{itemize}
\item Movimientos de la Tierra
 \begin{itemize}
  \item Traslación 
  \item Rotación
  \item Nutación
  \item Precesión 
 \end{itemize}
\item Solsticio y equinoccio
\item Distancias astronómicas 
 \begin{itemize}
 \item Unidades astronómicas
 \item Distancias en el Sistema Solar
 \item Paralaje
 \end{itemize} 
\item Propiedades de la luz
 \begin{itemize}
 \item Reflexión
 \item Refracción
 \item Dispersión 
 \item Difracción 
 \end{itemize}
\item Constelaciones 
 \begin{itemize}
  \item Historia (culturas)
  \item Orientación usando constelaciones 
  \item Carta celeste y apps
 \end{itemize}
\item Coordenadas 
 \begin{itemize}
  \item Coordenadas geográficas 
  \item Coordenadas Celestes
   \begin{itemize}
    \item Coordenadas horizontales
    \item Coordenadas ecuatoriales: horarias y absolutas
    \item Coordenadas eclipticas 
   \end{itemize}
 \end{itemize}
\end{itemize}


\section{Sesión 0}

\section{Sesión 1}
\end{document}